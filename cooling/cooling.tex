% -*- mode: latex; fill-column: 65; -*-
\documentclass[aps, pra, preprint]{revtex4-1}
\usepackage{amssymb}
\usepackage{amsmath}
\usepackage{siunitx}
\usepackage[colorinlistoftodos,prependcaption,textsize=tiny]{todonotes}
\usepackage{blindtext}
\presetkeys%
    {todonotes}%
    {inline,backgroundcolor=yellow}{}

\bibliographystyle{apsrev4-1}

\newcommand{\hfree}{H_{\rm free}}
\newcommand{\phicap}{\varphi_{\rm cap}}
\newcommand{\phiwall}{\varphi_{\rm wall}}


\begin{document}

\title{Computational analysis of laser cooling of ultra-cold ions
  in Penning traps
}

\author{Dominic Meiser}
\affiliation{Trimble Inc, Boulder, 4730 Walnut Street, Suite 201,
Boulder, CO 80301, USA.}
\author{John J Bollinger}
\affiliation{National Institute of Standards and Technology,
Boulder, Colorado 80305, USA.}

\begin{abstract}
  We develop a computational model for the analysis of ultra-cold
  ions in Penning traps. We extensively verify our finite
  temperature model by comparison with zero temperature models
  and by considering well understood limiting cases. We validate
  the model by comparison with experimental results for
  single-plane to multi-plane instabilities of the ion crystal
  and by comparison with experimentally observed spectra for the
  out-of-plane modes of the ion crystal.
  We then use our model to study the temperature of the in-plane
  modes which are experimentally challenging to access. We find
  that ...\todo{Findings for in-plane modes}
  \todo{Optimization of laser cooling parameters}
\end{abstract}

\maketitle


Ultra-cold ions in Penning traps enable studies at the forefront
of atomic physics\todo{references}, quantum
science\todo{references}, and condensed matter
physics\todo{references}. Many of these studies benefit from
lower temperatures. Colder temperatures increase the crystal
stability, decrease dephasing rates, and they can increase
interrogation times. Improved crystal stability has the potential
to enable optical access to individual lattice sites which could
lead to higher fidelity state preparation and state readout for quantum
metrology\todo{references} and quantum simulation
experiments\todo{references}.

The primary means of cooling ions in Penning traps are various
forms of laser cooling including Doppler cooling and side band
cooling. While the basic principles of laser cooling for
ultra-cold ions is the same as for neutral atoms in
magneto-optical traps there are also significant differences that
can make it challenging to understand the experimentally achieved
ion temperature, cooling limits, cooling and heating mechanisms,
and that can make it more challenging to optimize the cooling
laser parameters such as intensity and detuning as well as the
cooling laser geometry.

A major difference compared to neutral atom experiments is that
ions in Penning traps move in a magnetic field several Tesla
strong. The Lorentz force in the magnetic field forces the ions
into circular orbits with a cyclotron frequency on the order of a
few hundred kilo Hertz. The rotation of the ions in the magnetic
field means that the ions velocity periodically changes direction
relative to any cooling laser beams that are stationary in the
laboratory frame of reference. A second major difference is the
strong interaction between the ions due to the Coulomb force. The
Coulomb force couples the ions into collective modes of motion.
In contrast to the cooling dynamics of neutral atoms---which is
largely a single particle phenomenon---it is necessary to take
the collective nature of the ion motion into account to fully
understand the cooling dynamics of ultra-cold ions in Penning
traps.

To help us better understand the dynamics of ultra-cold ions in
Penning traps we have developed computer simulations that allow
us to quantitatively track the dynamics of the ions over time
scales from fractions of a nano-second all the way to milli
seconds and seconds. In this paper we demonstrate the validity of
these simulations by comparison with other simulations as well as
by comparison with experimental data. We then use these
simulations to study the spectra of the out-of-plane and in-plane
modes of motion of ultra-cold ions. We compute the temperatures
of these modes and determine optimal laser cooling parameters.

{\bf Who has studied cooling dynamics before? How did they do
it?} Need a significant review of the existing literature here.
Some key workds: Dan Dubin, Dave Wineland and JJ Bollinger,
Molecular Dynamics simulations, analytical studies, fluid models,
tracking codes, particle methods, PIC, Direct Simulation Mode
Carlo (DSMC).

The rest of this article is organized as follows. We begin by
discussing the mathematical model and computational approach
underlying our analysis in section~\ref{sec:model}. We then
verify our simulations by comparison with zero temperature models
in section~\ref{sec:verification}. Specifically, we compare
steady state solutions and  spectra for the out-of-plane modes of the
ion crystal. We validate the simulations by comparison with
experimental results for the single-plane to multi-plane
instability in section~\ref{sec:validation}. In
section~\ref{sec:inplanemodes} we consider the in-plane modes and
in section~\ref{sec:optimization} we present a numerical
optimization of the laser cooling parameters and we discuss the
coldest temperatures achievable with Doppler cooling. We wrap up
with section~\ref{sec:conclusion}


\section{Model and computational algorithm}
\label{sec:model}

In this section we describe our mathematical model as well as the
computational techniques used in our simulations.


\subsection{Mathematical model}

We treat the ions as classical point like particles with 
velocity $\mathbf{v}_i$ and position $\mathbf{x}_i$. The motion
of the ions is governed by the Hamiltonian
\begin{equation}
  H = \hfree + \sum_{i=1}^N q_i\varphi(\mathbf{x}_i)\;,
\end{equation}
where the free Hamiltonian
\begin{equation}
  \hfree =
  \sum_{i=1}^N \frac{1}{2m_i}\left(
    \mathbf{p}_i -
    q_i\mathbf{A}(\mathbf{x}_i) \right)^2 
  \label{eqn:total_hamiltonian}
\end{equation}
includes the vector potential $\mathbf{A}$ corresponding to the
axial magnetic field in the Penning trap, $\mathbf{B} =
\mathbf{\nabla}\times \mathbf{A}$, parallel to the $z$ axis. We
choose $\mathbf{A} = yB_z\mathbf{\hat x}$ with $\mathbf{\hat
  x}=\mathbf{x}/x$ the unit vector along $\mathbf{x}$. In
Eqn.~\eqref{eqn:total_hamiltonian} $N$ is the number of ions in
the trap, $m_i$ is the mass of ion $i$, $q_i$ its charge, and the
electrostatic potential
\begin{equation}
  \varphi(\mathbf{x}_i) =
  \phicap(\mathbf{x}_i) +
  \phiwall(\mathbf{x}_i) +
  \sum_{\substack{j=1\\j\neq i}}^N
  \frac{1}{4\pi\varepsilon_0}\
  \frac{q_j}{\left| \mathbf{x_i} - \mathbf{x_j} \right|}
\end{equation}
contains the potential $\phicap$ due to the end-cap electrodes in
the Penning trap, the rotating wall potential $\phiwall$, and the
Coulomb potential for the interaction between the ions. In the
vicinity of the ion crystal the end-cap and rotating wall
potentials are well approximated by harmonic potentials. We
parametrize them as
\begin{equation}
  \phicap(\mathbf{x}) +\phiwall(\mathbf{x}) =\frac{1}{2}k_z z^2 - 
\frac{1}{2} \left(k_x x_r^2 + k_y y_r^2\right)
\end{equation}
where
\begin{equation}
k_x=\left(\frac{1}{2}+\delta\right)k_z,\qquad 
k_y=\left(\frac{1}{2}-\delta\right)k_z\;,
\end{equation}
The dimensionless parameter $\delta$ characterizes the strength
of the rotating wall potential. The coordinates of the ions in
the rotating frame $[x_r, y_r]^T$ are given by
\begin{equation}
\left[
\begin{array}{c}
x_r\\
y_r
\end{array}\right] =
\left[
\begin{array}{cc}
\cos(\vartheta(t)) & -\sin(\vartheta(t))\\
\sin(\vartheta(t)) & \cos(\vartheta(t))
\end{array}\right]
\left[\begin{array}{c}
x\\
y
\end{array}\right]\;.
\end{equation}
The phase of the rotating wall potential is
\begin{equation}
\vartheta(t)=\omega_R t+\vartheta_0\;.
\end{equation}

In addition to the conservative dynamics described by the
Hamiltonian the ions are subject to radiation pressure forces due
the cooling lasers. We describe the radiation pressure force
using a stochastic model. We discuss this model in more detail
when we consider the computational integration of the ion motion
in the next section.


\subsection{Numerical algorithm}

To numerically integrate the motion of the ions we employ a split
step algorithm as is customary in e.g. molecular dynamics
simulations. To advance the positions and velocities of the ions
$\{\mathbf{x}_i, \mathbf{p}_i\}$
from time $t$ to $t + \Delta t$ we use the update formula
\begin{equation}
  \{\mathbf{x}_i, \mathbf{p}_i\}(t+\Delta t) =
  U_{\rm free}(\Delta t /2)
  U_{\rm kick}(t + \Delta t / 2; \Delta t)
  U_{\rm free}(\Delta t /2)
  \{\mathbf{x}_i, \mathbf{p}_i\}(t)\;.
\end{equation}
In this formula, $U_{\rm free}(\Delta t/2)$ is the time evolution
operator corresponding to the $\hfree$ that advance the state of
the ions for a time interval of duration $\Delta t / 2$. Since
the free Hamiltonian contains just the kinetic energy of the ions
and the Lorentz force due to the axial magnetic field the motion
generated by $U_{\rm free}$ is the well known circular Larmor
precession,
\todo{Formulas for drift along $z$ and circular motion in $x-y$
plane.}
\begin{eqnarray}
  \mathbf{x}_i &= \ldots\\
  \mathbf{p}_i &= \ldots
\end{eqnarray}
The time evolution operator $U_{\rm kick}(t + \Delta t/2; \Delta
t)$ corresponds to the interaction with the forces due to the
electrostatic potential as well as the radiation pressure forces
due to laser cooling. This operator is time dependent due to the
rotating wall potential and the radiation pressure forces. We
evaluate it at the mid-point of the time interval, $t + \Delta t
/ 2$. The operator $U_{\rm kick}$ changes the momenta of the
particles only,
\begin{eqnarray}
  \{\mathbf{x}_i, \mathbf{p}_i\}(t + \Delta t)
  &=& U_{\rm kick}(t + \Delta t /2; \Delta t)
    \{\mathbf{x}_i, \mathbf{p}_i\}(t)\\
  &=&\left\{
      \mathbf{x}_i(t),
      \mathbf{p}_i(t) +
      \Delta t q_i \mathbf{E}(t + \Delta t / 2, \mathbf{x}_i) +
      \Delta \mathbf{p}_i^{\rm laser}
      \right\}\;.
\end{eqnarray}
The kick due to the electric field is given by $\Delta t
q_i\mathbf{E}(t+\Delta t /2,\mathbf{x}_i) = -\Delta t
q_i\mathbf{\nabla}\varphi(\mathbf{x}_i)$.

The radiation pressure forces due to the laser cooling lead to
the momentum kick $\Delta \mathbf{p}_i^{\rm laser}$. To find this
kick we Find the number $n_{\rm scattered}$ of photons scattered
based on velocity and intensity at location of ion in time
interval $\Delta t$ (Doppler shift, saturated two level resonance
fluorescence, Poisson distributed)\todo{Discuss scattering
rates}, find $n_{\rm scattered}$ directions isotropically
distributed. This is only approximately correct as the scattering
pattern is not isotropic in reality. And then we find the
momentum kick as
\begin{equation}
\Delta \mathbf{p}_i^{\rm laser} = n_{\rm scattered}\hbar
\mathbf{k} + \sum_{i=1}^{n_{\rm scattered}} \hbar \mathbf{k}_i\;.
\end{equation}
\todo{Discuss limitations of this approach. Low saturation.}

\todo{Integrate the following section from previous draft into
  this discussion}
We adopt a very simplistic model of laser cooling which is directly
motivated by the microscopic physics upon which laser cooling based.
This model has the advantage of automatically including both
deterministic damping as well as the fluctuating forces inherent in
laser cooling.  Both of these components of the radiation pressure force
are required by the fluctuation dissipation theorem.  The balance
between them gives rise to the Doppler limit.

We model the internal structure of the ions by two quantum mechanical
levels representing the two states involved in the cycling transition
used for laser cooling.  A two level atom at position $\mathbf{r}$
scatters photons out of a laser beam with wavevector $\mathbf{k}$ with a
rate
\begin{equation}
\dot n (\mathbf{r}, \mathbf{v}) = 
S(\mathbf{r})\frac{\gamma_0}{2\pi}
\frac{(\gamma_0/2)^2}{(\gamma_0/2)^2(1+2S(\mathbf{r}))+\Delta^2(\mathbf{v})}\;,
\end{equation}
where $\gamma_0$ is the natural linewidth of the cycling transition (in
radians per second), $S(\mathbf{r})$ is the saturation parameter, and
$\Delta(\mathbf{v})=\Delta_0 + \mathbf{k}\cdot\mathbf{v}$ is the
detuning of the cooling transition from the laser frequency including
the first order Doppler shift.  We assume that the atoms scatter photons
with this rate with Poissonian number statistics\footnote{Strictly
speaking, this is only valid in the limit of low saturation (for
$S\sim 1$ we have to take into account anti-bunching).}.  We take into
account the beam profile by multiplying the saturation parameter with a
Gaussian factor that accounts for the spatial structure of the laser,
\begin{equation}
S(\mathbf{r})=e^{-\rho^2/\sigma^2}S_0\;,
\end{equation}
where $\rho$ is the distance of the atom from the center of the beam and
$\sigma$ is the $1/e$ radius of the intensity of the beam.

To simulate laser cooling we proceed as follows.  First we compute the
mean number of photons scattered by ion $j$ in time interval $\Delta t$,
\begin{equation}
n_j=\dot{n}_j \Delta t
\end{equation}
The velocities and positions needed for computing $\dot{n}_j$ are taken
at the center of the time step in accordance with the integration scheme
discussed above.  We then compute the actual number of photons scattered
by each ion as a Poissonian random number with mean $n_j$.  Each
particle receives a total momentum kick of 
\begin{equation}
\mathbf{\Delta p}_j = \mathbf{\Delta p_{j,{\rm absorb}}} + 
\mathbf{\Delta p_{j,{\rm emit}}}\;,
\end{equation}
where $\mathbf{\Delta p_{j,{\rm absorb}}}=n_j \hbar \mathbf{k}$ and
$\mathbf{\Delta p_{j,{\rm emit}}}$ is the recoil corresponging to $n_j$
photons scattered in random directions.  To compute $\mathbf{\Delta
p_{j,{\rm emit}}}$ we generate $n_j$ vectors of length $\hbar k$
pointing in random directions.  The recoil momentum $\mathbf{\Delta
p}_{j,{\rm emit}}$ is then obtained by adding up these vectors.

This approach captures the microscopic physics of laser cooling except
for two phenomena.  First, in the case of strong saturation,
$S\gtrsim 1$, quantum statistical phenomena start to play a role.
These manifest themselves in the form of anti-bunching of the photons
scattered in resonance fluorescence.  In essence, there is an
anti-correlation between photon emission events due to the fact that
immediately after a photon emission an atom is in the ground state with
certainty and therefore cannot emitt another photon right away.  This
correction to resonance fluorescence is probably rather unimportant in
the current context, especially because the saturation parameter is not
very much greater than 1.  The other approximation entering our model is
that the photon emission is nearly instantaneous relative to the
dynamics of the ions, i.e. the motion is uniform during an excited
state lifetime,
\begin{equation}
\eta \equiv T_{\rm cycl}(\gamma_0/(2\pi)) \ll 1\;.
\end{equation}
In our case this ratio is approximately $\eta \sim 0.3$.  Furthermore we
have to ensure in the simulations that
\begin{equation}
\Delta t \lesssim S\gamma_0/(2\pi)
\end{equation}
avoid having many photon emission events based on a velocity that was
only realized for a small fraction of the time step.


\section{Free space laser cooling}

As a simple test of our laser cooling model we simulate free space laser
cooling, i.e. the ions are not subject to any trapping or magnetic
fields.  Figure~\ref{fig:FreeSpaceCooling} shows the velocities of 10 Be
ions subject to laser cooling.  The atoms start with a velocity of
$20{\rm m}/{\rm s}$ in the positive $z$ direction.  Two counter
propagating lasers pointed along the $z$ axis irradiate the atoms. The
two counter propagating cooling lasers have peak intensities of $S=0.1$
and are detuned by $0.5\gamma_0$ to the red of the cycling transition at
$313{\rm nm}$.  The atomic linewidth is $\gamma_0=19\times {\rm MHz}$ .
The laser propagating in the $-z$ direction has a beam waist of
$0.01{\rm m}$ and the beam propagating in the $z$ direction has a
uniform intensity profile (i.e. inifinite beam diameter).

\todo{Make figure for free space cooling.}
As can be seen in Fig.~\ref{fig:FreeSpaceCooling}, the ions are cooled
to the Doppler limit.  Eventually until they diffuse out of the $-z$
beam.  After that they are accelerated by the unbalanced power in the
$z$ direction.  The rms velocity at the Doppler temperature is
\begin{equation}
\sqrt{\langle v_z^2\rangle} = \frac{\hbar \gamma_0}{3 m} \approx
0.25{\rm m}/{\rm s}\;,
\end{equation}
which is in good agreement with our simulations
\todo{Should measure our temperature and calibrate}.

In the absence of laser cooling the integration formula is
symplectic. This means that the integrator is long term stable
and has the qualitatively correct behavior. However, the laser
cooling terms break the symplecticity because the cooling force
depends both on the positions of the ions and their momenta.


\subsection{Convergence}

One of the most basic tests of the correctness of our time
integration scheme is to verify that it converges to a solution
as the time step size is reduced. To evaluate the convergence we
initialize our simulation with a steady state configuration of
127 ions shown in Fig.~\ref{fig:initial_state_top_view}. Here and
in the results discussed below we use parameters typical for the
Penning trap at NIST Boulder with a homogeneous magnetic field of
$B=\SI{4.4588}{\tesla}$,
a trap rotation frequency of $\omega_{\rm trap}=2\pi\times
\SI{180}{\kilo \hertz}$, end cap voltages yielding a confining
potential of $k_z=\ldots$\todo{What are the units here?}, and a
rotating wall potential of $V_{\rm Wall} = \SI{1}{\volt}$
yielding $\delta=\SI{3.5e-4}{}$.
\begin{figure}
  \includegraphics{./figures/fig_initial_state_top_view.pdf}
  \caption{Top view of steady state configuration of ions in
    Penning trap used for convergence study.}
  \label{fig:initial_state_top_view}
\end{figure}

Starting from this initial steady state configuration we
integrate the equations of motion for $\SI{10}{\us}$ using
different time step sizes. For each time step size we compare the
final solution with a reference solution computed using a time
step size of $\Delta t = \SI{2e-10}{\second}$. We compute the
error $\Delta x(\Delta t)$ in the simulation result
$\mathbf{x}(\Delta t)$ as the average Euclidian distance between
the reference solution $\mathbf{x}^{\rm
  ref}=\mathbf{x}(\SI{2.0e-10}{\second})$,
\begin{equation}
  \Delta x(\Delta t) =
  \sqrt{N^{-1}\sum_{i=1}^N\left(
      \mathbf{x}_i(\Delta t)-\mathbf{x}_i^{\rm ref}\right)^2}\;.
\end{equation}

The result is shown in Fig.~\ref{fig:convergence}. The error
decreases quadratically for time step sizes below approximately
$\SI{1.0e-7}{\second}$. At time step size of $\Delta t =
\SI{1}{\nano\second}$ the mean position error of the ions is
$\Delta x(\SI{1}{\nano\second})\approx \SI{1}{\nano\meter}$. For
reference, during this time interval the ions move on the order
of $\SI{1}{\milli\meter}$, i.e. the ion motion is integrated with
a relative error on the order of $\SI{1.0e-6}{}$. For time step
sizes greater than $\Delta t\approx 1.0e-7$ the integrator can be in
resonance with some of the vibrational eigen modes of the system.
Whenever the integrator is in resonance errors can grow
indefinitely. While damping forces such as laser cooling can
broaden the resonances and limit the growth of errors it is best
to stick to time step sizes that are smaller than $\Delta t =
\SI{1.0e-7}{\second}$. Results quoted in this paper were obtained
with $\Delta t = \SI{1}{\nano\second}$ unless noted otherwise.
\begin{figure}
  \includegraphics{./figures/fig_convergence.pdf}
  \caption{Integration errors as a function of time step size for
    a total time integration time of $T=\SI{10}{\us}$ for a
    typical Penning trap simulation. The dashed orange line is a
    quadratic for orientation. See text for details.}
  \label{fig:convergence}
\end{figure}


\section{Single-plane to multi-plane instability}
\label{sec:validation}


\section{Finite temperature mode analysis}

\subsection{Out-of-plane modes}

\subsection{In-plane modes}


\section{In-plane modes}
\label{sec:inplanemodes}


\section{Optimization of laser cooling parameters}
\label{sec:optimization}


\section{Conclusion}
\label{sec:conclusion}

\bibliography{cooling_bibliography}

\end{document}
