% -*- mode: latex; fill-column: 65; -*-
\documentclass[aps, pra, preprint]{revtex4-1}
\usepackage{todonotes}
\usepackage{amssymb}
\usepackage{amsmath}
\usepackage{siunitx}

\bibliographystyle{apsrev4-1}


\begin{document}

\title{Computational analysis of laser cooling of ultra-cold ions
  in Penning traps
  \todo{More imaginative title}
}

\author{Dominic Meiser\\
\todo{Fix author affiliations}
}
\affiliation{Timble Inc, ...}
\author{John J Bollinger}
\affiliation{NIST}

\begin{abstract}
  Here goes the abstract.
  \todo{Write abstract}
\end{abstract}

\maketitle


Ultra-cold ions in Penning traps are an experimental system that
enable many studies at the forefront of quantum science and
condensed matter physics. For many of these studies it is
beneficial to prepare the ultra-cold ion ensemble in the coldest
state possible. For example, for quantum
metrology\todo{references} and quantum simulation
experiments\todo{references}, cold temperatures enable single
site optical access to individual ions for precise state
preparation and high fidelity state readout. In squeezing
experiments colder temperatures can reduce dephasing rates.

{\bf For what types of experiments for are colder temperatures
beneficial?}
\begin{itemize}
\item Metrology and squeezing experiments
\item Quantum simulation experiments, Ising model, quantum
  magnetism, condensed matter, etc.
\item Quantum information and quantum computing? Is that within
  reach for Penning trap experiments?
\end{itemize}k

{\bf How do these experiments benefit from cold temperatures?}
\begin{itemize}
\item Improved crystal stability
\item Longer interrogation times
\item Higher fidelity state preparation
\item Smaller dephasing rates
\item Higher precision/fidelity state readout
\item single site access
\end{itemize}

{\bf How are cold temperatures achieved?} The primary means of
cooling ions in Penning traps are various forms of laser cooling
including Doppler cooling and side band cooling. While the basic
principles of laser cooling for ultra-cold ions is the same as
for neutral atoms in magneto-optical traps there are also
significant differences that can make it challenging to
understand the experimentally achieved ion temperature, cooling
limits, cooling and heating mechanisms, and that can make it more
challenging to optimize the cooling laser parameters such as
intensity and detuning as well as the cooling laser geometry.

{\bf What is different and more challenging for ions compared to
neutral atoms?} A major difference compared to neutral atom
experiments is that ions in Penning traps move in a magnetic
field several Tesla strong. The Lorentz force in the magnetic
field forces the ions into circular orbits with a cyclotron
frequency on the order of a few hundred kilo Hertz. The rotation
of the ions in the magnetic field means that the ions velocity
periodically changes direction relative to any cooling laser
beams that are stationary in the laboratory frame of reference. A
second major difference is the strong interaction between the
ions due to the Coulomb force. The Coulomb force couples the ions
into collective modes of motion. In contrast to the cooling
dynamics of neutral atoms---which is largely a single particle
phenomenon---it is necessary to take the collective nature of the
ion motion into account to fully understand the cooling dynamics
of ultra-cold ions in Penning traps.

{\bf What do we contribute in this paper?} To help us better
understand the dynamics of ultra-cold ions in Penning traps we
have developed computer simulations that allow us to
quantitatively track the dynamics of the ions over time scales
from fractions of a nano-second all the way to milli seconds and
seconds. In this paper we demonstrate the validity of these
simulations by comparison with other simulations as well as by
comparison with experimental data. We then use these simulations
to study the spectra of the out-of-plane and in-plane modes of
motion of ultra-cold ions. We compute the temperatures of these
modes and determine optimal laser cooling parameters.

{\bf Who has studied cooling dynamics before? How did they do
it?} Need a significant review of the existing literature here.
Some key workds: Dan Dubin, Dave Wineland and JJ Bollinger,
Molecular Dynamics simulations, analytical studies, fluid models,
tracking codes, particle methods, PIC, Direct Simulation Mode
Carlo (DSMC).

{\bf What will we discuss in the rest of this paper}


\section{Model and computational algorithm}


\subsection{Mathematical model}

\subsection{Numerical algorithm}


\subsection{Convergence}

\subsection{Single-plane to multi-plane instability}


\section{Finite temperature mode analysis}

\subsection{Out-of-plane modes}

\subsection{In-plane modes}



\section{Conclusion}

\bibliography

\end{document}
